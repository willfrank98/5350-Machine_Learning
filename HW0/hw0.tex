\documentclass[12pt, fullpage,letterpaper]{article}

\usepackage[margin=1in]{geometry}
\usepackage{url}
\usepackage{amsmath}
\usepackage{amssymb}
\usepackage{xspace}
\usepackage{graphicx}

\newcommand{\semester}{Spring 2019}
\newcommand{\assignmentId}{0}
\newcommand{\releaseDate}{7 January, 2019}
\newcommand{\dueDate}{11:59pm, 16 January, 2019}

\newcommand{\bx}{{\bf x}}
\newcommand{\bw}{{\bf w}}

\title{CS 5350/6350: Machine Learning \semester}
\author{Homework \assignmentId}
\date{Handed out: \releaseDate\\
  Due: \dueDate}

\begin{document}
\maketitle

% Math commands by Thomas Minka
\newcommand{\var}{{\rm var}}
\newcommand{\Tr}{^{\rm T}}
\newcommand{\vtrans}[2]{{#1}^{(#2)}}
\newcommand{\kron}{\otimes}
\newcommand{\schur}[2]{({#1} | {#2})}
\newcommand{\schurdet}[2]{\left| ({#1} | {#2}) \right|}
\newcommand{\had}{\circ}
\newcommand{\diag}{{\rm diag}}
\newcommand{\invdiag}{\diag^{-1}}
\newcommand{\rank}{{\rm rank}}
% careful: ``null'' is already a latex command
\newcommand{\nullsp}{{\rm null}}
\newcommand{\tr}{{\rm tr}}
\renewcommand{\vec}{{\rm vec}}
\newcommand{\vech}{{\rm vech}}
\renewcommand{\det}[1]{\left| #1 \right|}
\newcommand{\pdet}[1]{\left| #1 \right|_{+}}
\newcommand{\pinv}[1]{#1^{+}}
\newcommand{\erf}{{\rm erf}}
\newcommand{\hypergeom}[2]{{}_{#1}F_{#2}}

% boldface characters
\renewcommand{\a}{{\bf a}}
\renewcommand{\b}{{\bf b}}
\renewcommand{\c}{{\bf c}}
\renewcommand{\d}{{\rm d}}  % for derivatives
\newcommand{\e}{{\bf e}}
\newcommand{\f}{{\bf f}}
\newcommand{\g}{{\bf g}}
\newcommand{\h}{{\bf h}}
%\newcommand{\k}{{\bf k}}
% in Latex2e this must be renewcommand
\renewcommand{\k}{{\bf k}}
\newcommand{\m}{{\bf m}}
\newcommand{\mb}{{\bf m}}
\newcommand{\n}{{\bf n}}
\renewcommand{\o}{{\bf o}}
\newcommand{\p}{{\bf p}}
\newcommand{\q}{{\bf q}}
\renewcommand{\r}{{\bf r}}
\newcommand{\s}{{\bf s}}
\renewcommand{\t}{{\bf t}}
\renewcommand{\u}{{\bf u}}
\renewcommand{\v}{{\bf v}}
\newcommand{\w}{{\bf w}}
\newcommand{\x}{{\bf x}}
\newcommand{\y}{{\bf y}}
\newcommand{\z}{{\bf z}}
%s\newcommand{\l}{\boldsymbol{l}}
\newcommand{\A}{{\bf A}}
\newcommand{\B}{{\bf B}}
\newcommand{\C}{{\bf C}}
\newcommand{\D}{{\bf D}}
\newcommand{\E}{{\bf E}}
\newcommand{\F}{{\bf F}}
\newcommand{\G}{{\bf G}}
\renewcommand{\H}{{\bf H}}
\newcommand{\I}{{\bf I}}
\newcommand{\J}{{\bf J}}
\newcommand{\K}{{\bf K}}
\renewcommand{\L}{{\bf L}}
\newcommand{\M}{{\bf M}}
\newcommand{\N}{\mathcal{N}}  % for normal density
%\newcommand{\N}{{\bf N}}
\renewcommand{\O}{{\bf O}}
\renewcommand{\P}{{\bf P}}
\newcommand{\Q}{{\bf Q}}
\newcommand{\R}{{\bf R}}
\renewcommand{\S}{{\bf S}}
\newcommand{\T}{{\bf T}}
\newcommand{\U}{{\bf U}}
\newcommand{\V}{{\bf V}}
\newcommand{\W}{{\bf W}}
\newcommand{\X}{{\bf X}}
\newcommand{\Y}{{\bf Y}}
\newcommand{\Z}{{\bf Z}}

% this is for latex 2.09
% unfortunately, the result is slanted - use Latex2e instead
%\newcommand{\bfLambda}{\mbox{\boldmath$\Lambda$}}
% this is for Latex2e
\newcommand{\bfLambda}{\boldsymbol{\Lambda}}

% Yuan Qi's boldsymbol
\newcommand{\bsigma}{\boldsymbol{\sigma}}
\newcommand{\balpha}{\boldsymbol{\alpha}}
\newcommand{\bpsi}{\boldsymbol{\psi}}
\newcommand{\bphi}{\boldsymbol{\phi}}
\newcommand{\boldeta}{\boldsymbol{\eta}}
\newcommand{\Beta}{\boldsymbol{\eta}}
\newcommand{\btau}{\boldsymbol{\tau}}
\newcommand{\bvarphi}{\boldsymbol{\varphi}}
\newcommand{\bzeta}{\boldsymbol{\zeta}}

\newcommand{\blambda}{\boldsymbol{\lambda}}
\newcommand{\bLambda}{\mathbf{\Lambda}}
\newcommand{\bOmega}{\mathbf{\Omega}}
\newcommand{\bomega}{\mathbf{\omega}}
\newcommand{\bPi}{\mathbf{\Pi}}

\newcommand{\btheta}{\boldsymbol{\theta}}
\newcommand{\bpi}{\boldsymbol{\pi}}
\newcommand{\bxi}{\boldsymbol{\xi}}
\newcommand{\bSigma}{\boldsymbol{\Sigma}}

\newcommand{\bgamma}{\boldsymbol{\gamma}}
\newcommand{\bGamma}{\mathbf{\Gamma}}

\newcommand{\bmu}{\boldsymbol{\mu}}
\newcommand{\1}{{\bf 1}}
\newcommand{\0}{{\bf 0}}

% \newcommand{\comment}[1]{}

\newcommand{\bs}{\backslash}
\newcommand{\ben}{\begin{enumerate}}
\newcommand{\een}{\end{enumerate}}

 \newcommand{\notS}{{\backslash S}}
 \newcommand{\nots}{{\backslash s}}
 \newcommand{\noti}{{\backslash i}}
 \newcommand{\notj}{{\backslash j}}
 \newcommand{\nott}{\backslash t}
 \newcommand{\notone}{{\backslash 1}}
 \newcommand{\nottp}{\backslash t+1}
% \newcommand{\notz}{\backslash z}

\newcommand{\notk}{{^{\backslash k}}}
%\newcommand{\noti}{{^{\backslash i}}}
\newcommand{\notij}{{^{\backslash i,j}}}
\newcommand{\notg}{{^{\backslash g}}}
\newcommand{\wnoti}{{_{\w}^{\backslash i}}}
\newcommand{\wnotg}{{_{\w}^{\backslash g}}}
\newcommand{\vnotij}{{_{\v}^{\backslash i,j}}}
\newcommand{\vnotg}{{_{\v}^{\backslash g}}}
\newcommand{\half}{\frac{1}{2}}
\newcommand{\msgb}{m_{t \leftarrow t+1}}
\newcommand{\msgf}{m_{t \rightarrow t+1}}
\newcommand{\msgfp}{m_{t-1 \rightarrow t}}

\newcommand{\proj}[1]{{\rm proj}\negmedspace\left[#1\right]}
\newcommand{\argmin}{\operatornamewithlimits{argmin}}
\newcommand{\argmax}{\operatornamewithlimits{argmax}}

\newcommand{\dif}{\mathrm{d}}
\newcommand{\abs}[1]{\lvert#1\rvert}
\newcommand{\norm}[1]{\lVert#1\rVert}

%miscellaneous symbols
\newcommand{\ie}{{{i.e.,}}\xspace}
\newcommand{\eg}{{{\em e.g.,}}\xspace}
\newcommand{\EE}{\mathbb{E}}
\newcommand{\VV}{\mathbb{V}}
\newcommand{\sbr}[1]{\left[#1\right]}
\newcommand{\rbr}[1]{\left(#1\right)}
\newcommand{\cmt}[1]{}


\footnotesize
	\begin{itemize}
		\item You are welcome to talk to other members of the class about
		the homework. I am more concerned that you understand the
		underlying concepts. However, you should write down your own
		solution. Please keep the class collaboration policy in mind.
		
		\item Feel free discuss the homework with the instructor or the TAs.
		
		\item Your written solutions should be brief and clear. You need to
		show your work, not just the final answer, but you do \emph{not}
		need to write it in gory detail. Your assignment should be {\bf no
			more than 10 pages}. Every extra page will cost a point.
		
		\item Handwritten solutions will not be accepted.
		
		\item The homework is due by \textbf{midnight of the due date}. Please submit
		the homework on Canvas.
		
		\item Some questions are marked {\bf For 6350 students}. Students
		who are registered for CS 6350 should do these questions. Of
		course, if you are registered for CS 5350, you are welcome to do
		the question too, but you will not get any credit for it.
		
	\end{itemize}



\section*{Basic Knowledge Review}
\label{sec:q1}

%problem 1, decide wether dependency & indendency
%2, prove p(A + B) <= P(A) + P(B)
%3, prove P(\sum_i A_i) \le \sum_i p(A_i)
%4. given two a joint Gaussian, calculate the conditional Guassian distribution
%5. prove E(X) = E(E(X|Y))
%4, prove V(E(X)) = E(V(X))
%5. prove V(Y) = EV(Y|X) + VE(Y|X)

%independency, conditional distribution, expectation, variance, basic properties
%gradient calcualtion, logistic function, second derivatives
%
\begin{enumerate}
\item~[5 points] We use sets to represent events. For example, toss a fair coin $10$ times, and the event can be represented by the set of ``Heads" or ``Tails" after each tossing. Let a specific event $A$ be ``at least one head". Calculate the probability that event $A$ happens, i.e., $p(A)$.

$p(A) = 1 - \binom{10}{0}(.5)^0(.5)^{10} = 1 - 0.0009765625 = 0.9990234375$

\item~[10 points] Given two events $A$ and $B$, prove that 
\[
p(A \cup B) \le p(A) + p(B).
\]
When does the equality hold?

$p(A \cup B) = p(A) + p(B \cap A') $

$= p(A) + [p(B) - p(A \cap B)]$

$= p(A) + p(B) - p(A \cap B)$

and $p(A \cap B) \ge 0$

Therefore $p(A \cup B) \le p(A) + p(B)$

This is true for any two events that satisfy the basic properties of probability.

\item~[10 points] Let $\{A_1, \ldots, A_n\}$ be a collection of events. Show that
\[
p(\cup_{i=1}^n A_i) \le \sum_{i=1}^n p(A_i).
\]
When does the equality hold? (Hint: induction)

$p(\cup_{i=1}^n A_i) = p(A_1) + ... + p(A_n) - p(A_1 \cap A_2) - ... - p(\cap_{i=1}^n A_i)$

$p(\cup_{i=1}^n A_i) = \sum_{i=1}^n p(A_i) - p(A_1 \cap A_2) - ... - p(\cap_{i=1}^n A_i)$

Therefore $p(\cup_{i=1}^n A_i) \le \sum_{i=1}^n p(A_i).$

This equality will also hold for all events that satisfy the three axioms of probability.

%\item~[5 points] Given three events $A$, $B$ and $C$, show that
%\[
%p(A\cap B\cap C) = p(A|B\cap C)p(B|C)p(C)
%\]
\item~[20 points]  We use $\EE(\cdot)$ and $\VV(\cdot)$ to denote a random variable's mean (or expectation) and variance, respectively. Given two discrete random variables $X$ and $Y$, where $X \in \{0, 1\}$ and $Y \in \{0,1\}$. The joint probability $p(X,Y)$ is given in as follows:
\begin{table}[h]
        \centering
        \begin{tabular}{ccc}
        \hline\hline
         & $Y=0$ & $Y=1$ \\ \hline
         $X=0$ & $1/10$ & $2/10$ \\ \hline
         $X=1$  & $3/10$ & $4/10$ \\ \hline\hline
        \end{tabular}
        %\caption{Training data for the alien invasion problem.}\label{tb-alien-train}
        \end{table}
	
        \begin{enumerate}
            \item~[10 points] Calculate the following distributions and statistics. 
            \begin{enumerate}
            \item the the marginal distributions $p(X)$ and $p(Y)$
            
            \[
            p_X(x) = 
            \begin{cases} 
            	3/10 & x = 0 \\
            	7/10 & x = 1 \\ 
            	0 & otherwise \\
            \end{cases}
            \]
            
           	\[
            p_Y(y) = 
            \begin{cases} 
            	4/10 & y = 0 \\
            	6/10 & y = 1 \\ 
            	0 & otherwise \\
            \end{cases}
            \]
            
            \item the conditional distributions $p(X|Y)$ and $p(Y|X)$
            
            $P(A|B) = \frac{P(A \cap B)}{P(B)} $
            
            \[
            p(X|Y) = 
            \begin{cases} 
            1/4 & x = 0, y = 0 \\
            3/4 & x = 1, y = 0 \\ 
            1/3 & x = 0, y = 1 \\
            2/3 & x = 1, y = 1 \\
            0 & otherwise \\
            \end{cases}
            \]
            
            \[
            p(Y|X) = 
            \begin{cases} 
            	1/3 & x = 0, y = 0 \\
            	2/3 & x = 1, y = 0 \\ 
            	2/3 & x = 0, y = 1 \\
            	1/3 & x = 1, y = 1 \\
            	0 & otherwise \\
            \end{cases}
            \]
            \item $\EE(X)$, $\EE(Y)$, $\VV(X)$, $\VV(Y)$
            
            $\EE(X) = (0 * 3/10) + (1 * 7/10) = 7/10$
            
            $\EE(Y) = (0 * 4/10) + (1 * 6/10) = 6/10$
            
            $\VV(X) = (0^2 * 3/10) + (1^2 * 7/10) - (7/10)^2 = .21$
            
            $\VV(Y) = (0^2 * 4/10) + (1^2 * 6/10) - (6/10)^2 = .24$
            
            \item  $\EE(Y|X=0)$, $\EE(Y|X=1)$,  $\VV(Y|X=0)$, $\VV(Y|X=1)$
            
            $\EE(Y|X=0) = (0 * 1/3) + (1 * 2/3) = 2/3$
            
            $\EE(Y|X=1) = (0 * 2/3) + (1 * 1/3) = 1/3$
            
            $\VV(Y|X=0) = \EE((Y - \EE(Y|X=0))^2 | X=0) = \EE((Y - 2/3)^2 | X=0) = (4/9 * 1/3) + (1/9 * 2/3) = 0.074$
            
            $\VV(Y|X=1) = \EE((Y - \EE(Y|X=1))^2 | X=1) = \EE((Y - 1/3)^2 | X=1) = (1/9 * 1/3) + (4/9 * 2/3) = 0.296$
            
            \item  the covariance between $X$ and $Y$
            
            $Cov(X,Y) = (-7/10 * -6/10 * 1/10) + (-7/10 * 4/10 * 2/10) + (3/10 * -6/10 * 3/10) + (3/10 * 4/10 * 4/10) = -0.02$
            
            \end{enumerate}
            \item~[5 points] Are $X$ and $Y$ independent? Why?
            
            $X$ and $Y$ are not independent, this is because $p(x, y) \ne p_X(x) * p_Y(y)$ for all $(x, y)$ pairs. $p(0, 0) = 1/10, p_X(0) * p_Y(0) = 3/10 * 4/10 = 0.12$. 
            
            \item~[5 points] When $X$ is not assigned a specific value, are $\EE(Y|X)$ and $\VV(Y|X)$ still constant? Why?
            
            $\EE(Y|X)$ is a function of the random variable $X$, and is therefore a random variable itself. Additionally, $\VV(Y|X)$ is a function of $\EE(Y|X)$, making it a random variable as well.
            
        \end{enumerate}
\item~[10 points] Assume a random variable $X$ follows a standard normal distribution, \ie $X \sim \N(X|0, 1)$. Let $Y = e^X$. Calculate the mean and variance of $Y$.
\begin{enumerate}
	\item $\EE(Y)$
	
	$\EE(Y) = \int_{-\infty}^{\infty}\frac{1}{\sqrt{2\pi}}e^{-x^2/2} * e^x dx = \sqrt{e}$
	
	\item $\VV(Y)$
	
	$\VV(Y) = \int_{-\infty}^{\infty}(e^x - \sqrt{e})^2 * (\frac{1}{\sqrt{2\pi}}e^{-x^2/2})dx = e^2 - e$
\end{enumerate}

\item~[20 points]  Given two random variables $X$ and $Y$, show that 
\begin{enumerate}
\item $\EE(\EE(Y|X)) = \EE(Y)$

Let $g(x) = \EE(Y|X = x)$. Then $\EE(\EE(Y|X)) = \int_{-\infty}^{\infty}g(x)f_X(x)dx$.

Additionally, $g(x) = \int_{-\infty}^{\infty}y*f_{Y|X}(y|x)dy$, or $\int_{-\infty}^{\infty}y*\frac{f(x,y)}{f_X(x)}dy$.

Substituting this into the first integral gives $\int_{-\infty}^{\infty}\int_{-\infty}^{\infty}y*\frac{f(x,y)}{f_X(x)}*f_X(x)dydx$, 

or $\int_{-\infty}^{\infty}\int_{-\infty}^{\infty}y*f(x,y)dydx$, the definition of $\EE(Y)$.

\item$\VV(Y) = \EE(\VV(Y|X)) + \VV(\EE(Y|X))$

First we start by using the following formula for conditional variance:

$\VV(Y|X) =  \EE(Y^2|X) - [\EE(Y|X)]^2$

Taking the expectation of this gives $\EE(\VV(Y|X)) = \EE(\EE(Y^2|X)) - \EE([\EE(Y|X)]^2)$, which can be simplified to $\EE(Y^2) - \EE([\EE(Y|X)]^2)$.

Next we take the variance of conditional expected value, $\VV(\EE(Y|X)) = \EE([\EE(Y|X)]^2) - [\EE(\EE(Y|X))]^2$. Because $\EE(\EE(Y|X)) = \EE(Y)$, this can be simplified to $\EE([\EE(Y|X)]^2) - [\EE(Y)]^2$.

Finally these are combined, giving $\EE(\VV(Y|X)) + \VV(\EE(Y|X)) = \EE(Y^2) - \EE([\EE(Y|X)]^2) + \EE([\EE(Y|X)]^2) - [\EE(Y)]^2 = \EE(Y^2) - [\EE(Y)]^2 = \VV(Y)$.

Therefore $\VV(Y) = \EE(\VV(Y|X)) + \VV(\EE(Y|X))$

\end{enumerate}
(Hints: using definition.)

%\item~[20 points]  Let us go back to the coin tossing example. Suppose we toss a coin for $n$ times, \textit{independently}. Each toss we have $\frac{1}{2}$ chance to obtain the head. Let us denote the total number of heads by $c(n)$. Derive the following statistics. You don't need to give the numerical values. You only need to provide the formula.
%\begin{enumerate}
%\item $\EE(c(1))$, $\VV(c(1))$
%\item $\EE(c(10))$, $\VV(c(10))$
%\item $\EE(c(n))$, $\VV(c(n))$
%\end{enumerate} 
%What can you conclude from comparing the expectations and variances with different choices of $n$?  

\item~[15 points] Given a logistic function, $f(\x) = 1/(1+\exp(-\a^\top \x))$ ($\x$ is a vector), derive/calculate the following gradients and Hessian matrices.  
\begin{enumerate}
\item $\nabla f(\x)$

$f(\x) = 1/(1+\e^{-\a_1\x_1 - ... - \a_n\x_n})$

$\nabla f(\x) =
\begin{Bmatrix}
	-\frac{1+\e^{-\a_1\x_1 - ... - \a_n\x_n}}{(1+\e^{-\a_1\x_1 - ... - \a_n\x_n})^2} \\
	\vdots \\
	-\frac{1+\e^{-\a_1\x_1 - ... - \a_n\x_n}}{(1+\e^{-\a_1\x_1 - ... - \a_n\x_n})^2} \\
\end{Bmatrix}$

\item $\nabla^2 f(\x)$

$
\nabla^2 f(\x) = 
\begin{Bmatrix}
	\frac{1+\e^{-\a_1\x_1 - ... - \a_n\x_n}}{(1+\e^{-\a_1\x_1 - ... - \a_n\x_n})^2} & \dots &  \frac{1+\e^{-\a_1\x_1 - ... - \a_n\x_n}}{(1+\e^{-\a_1\x_1 - ... - \a_n\x_n})^2} \\
	\vdots & & \vdots\\
	\frac{1+\e^{-\a_1\x_1 - ... - \a_n\x_n}}{(1+\e^{-\a_1\x_1 - ... - \a_n\x_n})^2} & \dots & \frac{1+\e^{-\a_1\x_1 - ... - \a_n\x_n}}{(1+\e^{-\a_1\x_1 - ... - \a_n\x_n})^2}\\
\end{Bmatrix}
$

\item $\nabla f(\x)$ when $\a = [1,1,1,1,1]^\top$ and $\x = [0,0,0,0,0]^\top$

$
\begin{Bmatrix}
	-.5 \\
	-.5 \\
	-.5 \\
	-.5 \\
	-.5 \\
\end{Bmatrix}
$

\item $\nabla^2 f(\x)$  when $\a = [1,1,1,1,1]^\top$ and $\x = [0,0,0,0,0]^\top$

$
\begin{Bmatrix}
.5 & .5 & .5 & .5 & .5 \\
.5 & .5 & .5 & .5 & .5 \\
.5 & .5 & .5 & .5 & .5 \\
.5 & .5 & .5 & .5 & .5 \\
.5 & .5 & .5 & .5 & .5 \\
\end{Bmatrix}
$

\end{enumerate}
Note that $0 \le f(\x) \le 1$.

\item~[10 points] Show that $g(x) = -\log(f(\x))$ where $f(\x)$ is a logistic function defined as above, is convex. 

$\A = \nabla^2 g(x) = 
\begin{Bmatrix}
\frac{1+\e^{-\a_1\x_1 - ... - \a_n\x_n}}{ln(10)(1+\e^{-\a_1\x_1 - ... - \a_n\x_n})^2} & \dots &  \frac{1+\e^{-\a_1\x_1 - ... - \a_n\x_n}}{ln(10)(1+\e^{-\a_1\x_1 - ... - \a_n\x_n})^2} \\
\vdots & & \vdots\\
\frac{1+\e^{-\a_1\x_1 - ... - \a_n\x_n}}{ln(10)(1+\e^{-\a_1\x_1 - ... - \a_n\x_n})^2} & \dots & \frac{1+\e^{-\a_1\x_1 - ... - \a_n\x_n}}{ln(10)(1+\e^{-\a_1\x_1 - ... - \a_n\x_n})^2}\\
\end{Bmatrix}
$

For $g(x)$ to be convex, then $\v^T\A\v$ must be positive for all non-negative $\v \in \mathbb{R}^n$, aka semi-definite positive. 

All the second partial derivatives in the Hessian of $g(x)$ are strictly positive, therefore $\v^T\A\v$ must always be positive, and $g(x)$ must be convex. 

\end{enumerate}


\end{document}
%%% Local Variables:
%%% mode: latex
%%% TeX-master: t
%%% End:
